\section{Fault Model}	\label{sec:fault_model}
The Fault Model shows the type and nature of a fault. Depending upon the impact of the fault, a fault model can be of bit level or byte model or, word level. Also, it can be categorized by seeing whether its distribution is unform or random.
	The attack we present here falls under the random nibble level fault model. After choosing the nibble randomly we give fault to generate some ciphertexts and try to recover the key from there.


\subsection{Fault Invarient from Division Property}		\label{subsec:fault_invarient_from_division_property}
Fault attacks are most useful in practical scenario. If an attacker gets access of the device, he can manipulate with the device and get the hidden information from there. Within the fault models, the most powerful is the random distribution one. In that case the attacker can fault on any random position and of that only he can exploit the cipher. 
	In this paper, our attack is based upon random nibble fault model. Hence the distribution we are taking at the time of giving fault is random and the attack is on nibble level. Initially we choose a nibble randomly from the cipher and give faults to generate an input Division Property at that position. The property then propagates through the cipher and generates Division Trials. As described earlier if we get the output Division Trail variables as 0. We get balanced bits in those position. In this work, we show for these two SPN structures \present and \gift we always get the balanced bits at the same position irrespective of giving the initial input Division Property at any random nibble. This seems the propagation of the All property that we generally do in Integral Cryptanalysis but the main difference is that here we can give All at any random nibble and in every case we get balanced bits at same position. This is a fault invariant model for the SPN structures \present and \gift as after choosing the nibble randomly also the balanced bit remains at constant position. 

%photo --> giving fault, time to propagate, decrypting round


\subsection{Results using Division Property on \present and \gift  }		\label{subsec:result_on_present_gift}
As described earlier, we use Division Property to find balanced position after some rounds in the cipher. For \present we get a 5-round property of the cipher. We give a initial input Division Property at a nibble and depending upon the propagation of the Division Trail after 5 rounds we get balanced positionsin some particular bits. After the permutation layer of the 5th round, the balanced bits occurs at the first 4 nibbles. Hence we are getting total 16 bit balanced positions which remains invariant if we give input Division Property at random nibble. 
	Like \present, for \gift-64 also we get the random nibble property for 5 rounds. Here the balanced bits are at different position than \present. For \gift-64 we get 16 bits which appears in the 0th bit of each nibble after the permutation layer of 5th round.
	Like the previous two ciphers, for \gift-128 we get some balanced bit positions after 5 rounds. Interestingly for \gift-128 we get two sets of balanced bits, eachconsists of 80 bit postion. The first set occurs if we give the input Division Property at nibbles 0 to 15. The other set occurs if we give the initial input Division Property at the nibble postions 16 to 31. The two sets have an intersection of .... bit positions that we get in \autoref{...}.



\begin{center}
	\begin{table}[!h]
		\begin{Tabular}[1.3]{|c|c|c|c|}
			\hline
			round & \# active bits & \#balanced bits & balanced bits position	\\
			\hline
			5 & 4 & 4 & 0,1,2,3	\\
			5 & 3 & 0 &		\\
			\hline
			4 & 4 & 64 & 0(1)64	\\
			4 & $3^\ast$ & 16 & 0(1)16	\\
			\hline
			3 & $2^\ast$ & 16 & 0, 4, 8, 12, 16, 20, 24, 28, 32, 36, 40, 44, 48, 52, 56, 60	\\
			3 & $2^\ast$ &16 &2, 6, 10, 14, 18, 22, 26, 30, 34, 38, 42, 46, 50, 54, 58, 62	\\
			3 & 1 & 0 &	\\ 
			\hline
		\end{Tabular}
		\caption{balanced bit position after last p-layer of \present-64}
		\label{tab:present_balaned_bits}		
	\end{table}
\end{center}



\begin{center}
	\begin{table}[!h]
		\begin{Tabular}[1.3]{|c|c|c|c|}
			\hline
			round & \# active bits & \#balanced bits & balanced bits position	\\
			\hline
			5 & 4 & 16 & 0, 4, 8, 12, 16, 20, 24, 28, 32, 36, 40, 44, 48, 52, 56, 60$\ast$	\\
			5 & 3 & 0 &		\\
			\hline
			4 & 4 & 64 & 0(1)64	\\
			4 & $3^\ast$ & 24 & $\ast$	\\
			4 & 2 & 0 &		\\			
			\hline
			3 & $2^\ast$ &  & $\ast$	\\
			3 & 1 & 0 & 0	\\
			\hline
		\end{Tabular}
		\caption{balanced bit position after last p-layer of \gift--64}
		\label{tab:gift_balaned_bits}		
	\end{table}
\end{center}




